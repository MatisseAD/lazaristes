\documentclass[11pt,a4paper]{article}
\usepackage[utf8]{inputenc}
\usepackage[T1]{fontenc}
\usepackage[french]{babel}
\usepackage{amsmath,amssymb,amsthm}
\usepackage{geometry}
\geometry{margin=1.8cm}
\setlength{\parindent}{0pt}

\usepackage[dvipsnames]{xcolor}
\usepackage[most]{tcolorbox}

% --- couleurs globales fiche ---
\definecolor{FicheBlue}{RGB}{33, 102, 172}
\definecolor{FicheGreen}{RGB}{44, 162, 95}
\definecolor{FicheOrange}{RGB}{253, 174, 97}
\definecolor{FicheGrey}{RGB}{240, 240, 240}

% --- styles d'encadrés ---
\tcbset{
  colback=FicheGrey,
  colframe=FicheBlue,
  coltitle=black,
  boxrule=0.8pt,
  arc=3mm,
  left=4pt,right=4pt,top=3pt,bottom=3pt,
  fonttitle=\bfseries,
}

\newtcolorbox{defbox}[1][]{title={\textbf{Définition}},colback=blue!3,#1}
\newtcolorbox{propbox}[1][]{title={\textbf{Propriétés}},colback=FicheGrey,#1}
\newtcolorbox{exbox}[1][]{title={\textbf{Exemples}},colback=green!5!white,colframe=FicheGreen,#1}
\newtcolorbox{methbox}[1][]{title={\textbf{Mémo méthode}},colback=FicheOrange!12!white,colframe=FicheOrange,#1}
\newtcolorbox{titlebox}[1][]{enhanced,sharp corners,boxrule=0pt,colback=FicheBlue!15!white,#1}

\newcommand{\R}{\mathbb{R}}
\newcommand{\C}{\mathbb{C}}
\newcommand{\K}{\mathbb{K}}
\newcommand{\Id}{\mathrm{Id}}

\begin{document}

\begin{titlebox}
\centering
{\Large \textbf{Fiche -- Espaces vectoriels}}\\[0.3em]
{Fait avec amour et passion par votre homme Matisse}
\end{titlebox}

\vspace{0.4cm}

%------------------------------------------------
\section*{\textcolor{FicheBlue}{1. Espace vectoriel}}

\begin{defbox}
\textbf{Donnée :} un corps $\K$.

Un \textbf{$\K$-espace vectoriel} est un triplet $(E,+,\cdot)$ où :
\begin{itemize}
  \item $(E,+)$ est muni de l'associativité, commutativité, élément neutre $0$, opposé $-x$.
  \item $\cdot : \K \times E \to E$ vérifie, pour tout $x,y\in E$ et $\lambda,\mu\in\K$ :
  \[
  \lambda(x+y)=\lambda x+\lambda y,\quad
  (\lambda+\mu)x = \lambda x + \mu x,\quad
  \lambda(\mu x)=(\lambda\mu)x,\quad
  1x = x.
  \]
\end{itemize}
\end{defbox}

\begin{propbox}
Pour tout $x\in E$, $\lambda\in\K$ :
\[
0x = 0,\quad \lambda 0 = 0,\quad (-\lambda)x = \lambda(-x) = -(\lambda x),\quad
\lambda x = 0 \Rightarrow (\lambda = 0 \ \text{ou}\ x = 0).
\]
\end{propbox}

\begin{exbox}
Exemples d'espaces vectoriels :
\[
\K^n,\quad \R^{\mathbb{N}}\ (\text{suites}),\quad F(X,\K),\quad
\text{espaces de polynômes},\quad \text{espaces de matrices}.
\]
\end{exbox}

%------------------------------------------------
\section*{\textcolor{FicheBlue}{2. Sous-espaces vectoriels et sous-espace engendré}}

\begin{defbox}
$F\subset E$ est un \textbf{sous-espace vectoriel} (sev) de $E$ si
\[
0\in F \quad\text{et}\quad
\forall x,y\in F,\ \forall\lambda,\mu\in\K,\ \lambda x+\mu y\in F.
\]
\end{defbox}

\begin{methbox}
\textbf{Pour montrer que $F$ est un sev de $E$ :}
\[
0\in F,\quad
x,y\in F\Rightarrow x+y\in F,\quad
x\in F,\lambda\in\K\Rightarrow \lambda x\in F.
\]
\end{methbox}

\begin{propbox}
\begin{itemize}
  \item Intersection d'une famille de sev $\Rightarrow$ sev.
  \item En général, $F\cup G$ n'est \textbf{pas} un sev.
\end{itemize}
\end{propbox}

\begin{defbox}
Pour $A\subset E$, le \textbf{sous-espace engendré} par $A$, noté $\mathrm{Vect}(A)$, est le plus petit sev contenant $A$.

Si $A=\{x_1,\dots,x_n\}$ :
\[
\mathrm{Vect}(x_1,\dots,x_n)
= \{\lambda_1x_1+\cdots+\lambda_n x_n : \lambda_i\in\K\}
= \mathrm{Vect}(A).
\]
\end{defbox}

\begin{exbox}
\textbf{Droite vectorielle.} Pour $x\in E$,
\[
\mathrm{Vect}(x)=\K x=\{\lambda x:\lambda\in\K\}.
\]
Si $E=\K x$ pour un $x\neq 0$, alors $E$ est une \textbf{droite vectorielle}.
\end{exbox}

\begin{propbox}
\textbf{Colinéarité.} 

$x,y\in E$ sont \textbf{colinéaires} si $\exists\lambda\in\K$ tel que $y=\lambda x$ (ou $x=\lambda y$).
\end{propbox}

%------------------------------------------------
\section*{\textcolor{FicheBlue}{3. Applications linéaires, noyau, image}}

Soient $E,F$ deux $\K$-espaces vectoriels.

\begin{defbox}
$f:E\to F$ est \textbf{linéaire} si
\[
\forall x,y\in E,\ \forall\lambda,\mu\in\K,\quad
f(\lambda x+\mu y)=\lambda f(x)+\mu f(y).
\]
On note $L(E,F)$ l'ensemble des applications linéaires de $E$ dans $F$, et $L(E)=L(E,E)$.
\end{defbox}

\begin{methbox}
    Pour montrer qu'une application f est linéaire, il faut montrer que :
    \[
    \forall x,y\in E, \ \forall\lambda,\mu\in\K \ f(\lambda x \ +  \mu y) = \lambda f(x)  \ + \ \mu f(y)
    \]
\end{methbox}

\begin{exbox}
Cas particuliers :
\begin{itemize}
  \item $E=F$ : $f$ est un \textbf{endomorphisme}.
  \item $f$ bijective : \textbf{isomorphisme}.
  \item Endomorphisme + bijectif : \textbf{automorphisme}.
\end{itemize}
\end{exbox}

\begin{propbox}
\textbf{Noyau et image.}
\[
\ker f = \{x\in E : f(x)=0\},\qquad
\mathrm{Im}\,f = \{f(x): x\in E\}.
\]
Ce sont des sous-espaces vectoriels.

\medskip
\textbf{Lien avec injectivité/surjectivité :}
\[
f\ \text{injective} \Longleftrightarrow \ker f = \{0\},
\qquad
f\ \text{surjective} \Longleftrightarrow \mathrm{Im}\,f = F.
\]
\end{propbox}

\begin{propbox}
\[
\forall f,g \in GL(E), \ f \circ g \in GL(E)
\]
\end{propbox}

%------------------------------------------------
\section*{\textcolor{FicheBlue}{4. L(E) et groupe linéaire GL(E)}}

\begin{defbox}
\textbf{Espace vectoriel $L(E,F)$.} Avec les opérations
\[
(f+g)(x)=f(x)+g(x),\qquad
(\lambda f)(x)=\lambda f(x),
\]
$L(E,F)$ est un $\K$-espace vectoriel.
\end{defbox}

\begin{defbox}
\textbf{Groupe linéaire.} On note
\[
GL(E)=\{u\in L(E)\ :\ u \text{ est bijective}\}.
\]
Avec la composition :
\begin{itemize}
  \item $\Id_E\in GL(E)$ ;
  \item $u,v\in GL(E)\Rightarrow v\circ u\in GL(E)$ ;
  \item $u\in GL(E)\Rightarrow u^{-1}\in GL(E)$.
\end{itemize}
\end{defbox}

%------------------------------------------------
\section*{\textcolor{FicheBlue}{5. Somme, somme directe, supplémentaires}}

\begin{defbox}
Soit $E$ un $\K$-espace vectoriel, $A,B$ deux sous-espaces vectoriels.

\textbf{Somme.}
\[
A+B=\{a+b : a\in A,\ b\in B\},
\]
c'est le plus petit sev contenant $A$ et $B$.
\end{defbox}

\begin{defbox}
\textbf{Somme directe.} On dit que $A$ et $B$ sont en \textbf{somme directe} si
\[
\forall a\in A,\forall b\in B,\ a+b=0 \Rightarrow a=b=0.
\]
On note alors $A\oplus B$ au lieu de $A+B$.
\end{defbox}

\begin{methbox}
\textbf{Somme direct :}
\[
A\oplus B \Longleftrightarrow A\cap B=\{0\}.
\]
\end{methbox}

\begin{defbox}
\textbf{Sous-espaces supplémentaires.} $A$ et $B$ sont \textbf{supplémentaires} dans $E$ si
\[
E=A\oplus B.
\]
Autrement dit, tout $x\in E$ s’écrit d’une \emph{unique} façon $x=a+b$ avec $a\in A$, $b\in B$. 
Pour montrer une telle égalité, l'analyse synthèse est une méthode pertinente
\end{defbox}

\begin{methbox}
\textbf{Version géométrique du théorème du rang.}\\[0.2em]
Soit $f\in L(E,F)$ et $A$ un supplémentaire de $\ker f$ dans $E$ : $E=\ker f\oplus A$.
Alors l’application
\[
\varphi:A\to \mathrm{Im}\,f,\quad x\mapsto f(x)
\]
est un isomorphisme.
\end{methbox}

%------------------------------------------------
\section*{\textcolor{FicheBlue}{6. Projecteurs}}

\begin{defbox}
Un \textbf{projecteur} de A parallèlement à B est un endomorphisme $p\in L(E)$ tel que
\[
p^2 = p.
\
\forall (a,b) \in (A,B), \ p(a+b) = a
\]
\end{defbox}

\begin{propbox}
\textbf{Lien avec les supplémentaires.}\\[0.2em]
Soit $A,B$ deux sev tels que $E=A\oplus B$. Alors il existe un \textbf{unique} projecteur $p$ tel que
\[
\mathrm{Im}\,p = A,\quad \ker p = B,
\]
et pour tout $x=a+b$ (avec $a\in A$, $b\in B$),
\[
p(x)=a.
\]

Réciproquement, si $p$ est un projecteur, alors
\[
E=\ker p \oplus \mathrm{Im}\,p.
\]
\end{propbox}

%------------------------------------------------
\section*{\textcolor{FicheBlue}{7. Symétries}}

\begin{defbox}
Soit $A,B$ deux sev supplémentaires de $E$ : $E=A\oplus B$.
La \textbf{symétrie par rapport à $A$ parallèlement à $B$} est l’endomorphisme $s\in L(E)$ défini par
\[
\forall a\in A,\forall b\in B,\quad s(a+b)=a-b.
\]
\end{defbox}

\begin{propbox}
\begin{itemize}
  \item $s\circ s = \Id_E$ donc $s$ est un automorphisme et $s^{-1}=s$.
  \item $\ker(s-\Id) = A$, $\ker(s+\Id) = B$.
  \item Une application $s\in L(E)$ est une symétrie $\Longleftrightarrow s^2=\Id_E$.
\end{itemize}
\end{propbox}

%------------------------------------------------
\section*{\textcolor{FicheBlue}{8. Hyperplans}}

\begin{defbox}
Un \textbf{hyperplan} de $E$ est le noyau d'une forme linéaire non nulle
\[
H = \ker \varphi, \quad \varphi\in E^\star=L(E,\K),\ \varphi\neq 0.
\]
\end{defbox}

\begin{propbox}
\begin{itemize}
  \item Un hyperplan est un sev strict de $E$.
  \item $H$ est un hyperplan $\Longleftrightarrow$ $H$ admet une \textbf{droite vectorielle} comme supplémentaire :
  \[
  \exists x_0\notin H\quad E = H \oplus \K x_0.
  \]
  \item Si $E=\K^n$, les hyperplans sont les ensembles
  \[
  H = \{(x_1,\dots,x_n)\in\K^n : a_1x_1+\cdots+a_n x_n = 0\},
  \]
  avec $(a_1,\dots,a_n)\neq 0$ (droites par l’origine dans $\R^2$, plans par l’origine dans $\R^3$, etc.).
\end{itemize}
\end{propbox}

\end{document}
